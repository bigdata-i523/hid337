\documentclass[sigconf]{acmart}

\usepackage{hyperref}

\usepackage{endfloat}
\renewcommand{\efloatseparator}{\mbox{}} % no new page between figures

\usepackage{booktabs} % For formal tables

\settopmatter{printacmref=false} % Removes citation information below abstract
\renewcommand\footnotetextcopyrightpermission[1]{} % removes footnote with conference information in first column
\pagestyle{plain} % removes running headers

\begin{document}
\title{Big Data and Artificial Intelligence Solutions for in Home, Community and Territory Security}


\author{Ashok Reddy Singam}
\orcid{HID337}
\affiliation{%
  \institution{Indiana University}
  \streetaddress{711 N Park Ave}
  \city{Bloomington} 
  \state{Indiana} 
  \postcode{47408}
}
\email{asingam@iu.edu}

\author{Anil Ravi}
\orcid{HID333}
\affiliation{%
  \institution{Indiana University}
  \streetaddress{711 N Park Ave}
  \city{Bloomington} 
  \state{Indiana} 
  \postcode{47408}
}
\email{anilravi@iu.edu}

\begin{abstract}
Having an intelligent ear-and-eye monitoring system at the home to constantly observe the surroundings both inside and outside can protect the house and personnel much more safer way. By extending this capability to the neighborhood and city through collaboration would create safe cities across the world. Here we are going to review existing systems, methods and research literature on security at various geographical levels to propose the improved systems implemented with artificial intelligence and big data infrastructure. Video surveillance of residential, commercial, military, and other restricted locations have been in practice since many years using various available technologies. Depending on the level of security, the data has been processed by data mining and/or big data analytics to take decisions by various personal, agencies and governments. However, the limitations of data collection, data mining and adoption of intelligence led to ineffective systems which are not predictive as they should be. Here we are proposing to use the integrated video and audio data of targeted regions (homes, public places and extended areas) along with social media data for analysis. The system uses advanced statistical methods, machine learning and classification algorithms to predict and prevent the threats based on the severity probability.

\end{abstract}

\keywords{i523, HID333, HID337, Artificial Intelligence, Natural Language Processing (NLP), Machine Learning, Micro Drone.}


\maketitle

\section{Introduction}

In the today's technology world, drones are becoming more familiar in the main stream life activities such as recreational, spy cameras by authorities, home delivery of goods etc.
The present security systems used by households are static cameras used at a fixed location inside or outside the house. They are connected to network and provide alerts when any event occurred. However, they are not intelligent enough to understand the context, recognizing the people faces, and aware of family members behaviors, house needs etc.  By making cameras movable across the house and outside and process the data as humans do and take decisions of alerting or informing to the right people is the key concept of this paper.This system to be functional, the following technologies needed:

\begin{itemize}
  
\item Micro drones with audio and video sensors\cite{Labs2016}

\item Facial recognition and mapping through video analytics to recognize people
	
\item Natural language processing (NLP) to recognize family members, friends and strangers

\item Machine learning algorithms to understand family members habits and behaviors

\item Interfacing with email servers, phone, text message servers

\end{itemize}

\subsection{Contemporary Security Systems}

\subsection{Dynamic Video Analytics}

Video analytics plays a key role in modernizing video surveillance systems.Deep Learning the fastest growing field of Artificial Intelligence, enabling computers to interpret huge 
amounts of data like videos.The Graphics Processing Units (GPUs) provided 
by vendors like Nvidia enabling the deep learning infrastructure to cameras and recorders. 

\subsection{Live Voice Analytics}

\section{Big Data Infrastructure}

\section{Home Security}

\section{Community/Regional Security}

\section{Extended Regional Security}

\section{Conclusions}

\begin{acks}
The authors would like to thank professor Gregor von Laszewski and his team for providing LaTex templates
\end{acks}

\bibliographystyle{ACM-Reference-Format}
\bibliography{report} 

\end{document}
