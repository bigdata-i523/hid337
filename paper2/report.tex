\documentclass[sigconf]{acmart}

\usepackage{graphicx}
\usepackage{hyperref}
\usepackage{todonotes}

\usepackage{endfloat}
\renewcommand{\efloatseparator}{\mbox{}} % no new page between figures

\usepackage{booktabs} % For formal tables

\settopmatter{printacmref=false} % Removes citation information below abstract
\renewcommand\footnotetextcopyrightpermission[1]{} % removes footnote with conference information in first column
\pagestyle{plain} % removes running headers

\newcommand{\TODO}[1]{\todo[inline]{#1}}

\begin{document}
\title{Natural Language Processing (NLP) to analyze human speech data}

\author{Ashok Reddy Singam}
\orcid{HID337}
\affiliation{%
  \institution{Indiana University}
  \streetaddress{711 N Park Ave}
  \city{Bloomington} 
  \state{Indiana} 
  \postcode{47408}
}
\email{asingam@iu.edu}

\author{Anil Ravi}
\orcid{HID333}
\affiliation{%
  \institution{Indiana University}
  \streetaddress{711 N Park Ave}
  \city{Bloomington} 
  \state{Indiana} 
  \postcode{47408}
}
\email{anilravi@iu.edu}

\begin{abstract}
Extracting meaningful information from large volumes of unstructured human language is a challenging big data problem. Automatic speech recognition (ASR) and natural language processing (NLP) based intelligent system can be  used in several human machine interface applications both in consumer and industrial sector. Here describing the architecture, building blocks, performance and applications for such system that would use pre-developed ASR and NLP APIs.
\end{abstract}

\keywords{i523, HID333, HID337, Natural Language Processing}


\maketitle


\section{Introduction}
As voice becoming a common user interface, the need for accurate and intelligent speech recognition technologies is growing. In speech processing technology there are two main subtasks 
   \begin{itemize}
     \item Speaker Recognition
     \item Speech Recognition
   \end{itemize}
Although the performance of current speaker and speech  recognition systems is far from perfect, these  systems  have already proven their usefulness for certain applications. 
\subsection{SPEAKER RECOGNITION}
Speaker identification is one of the important task in speech processing. Each person has a voice that is different from everyone else’s.Speaker recognition is the process of identifying who is speaking by using acoustic features of speech.Speaker recognition has been applied mostly in security applications to control access. Current speaker recognition systems are not very accurate for large speaker populations. 

\subsection{SPEECH RECOGNITION}
Speech recognition is the ability to identify spoken words.Modern speech recognition research began in the late 1950s with the advent of the digital computer. The 1960s saw advances in the automatic segmentation of speech into units of linguistic relevance like phonemes, syllables. And now pattern matching and classification algorithms.


\section{Conclusion}



\begin{acks}

  The authors would like to thank Dr. Gregor von Laszewski for his support and suggestions to write this paper.

\end{acks}

\bibliographystyle{ACM-Reference-Format}
\bibliography{report} 

\appendix



\end{document}
